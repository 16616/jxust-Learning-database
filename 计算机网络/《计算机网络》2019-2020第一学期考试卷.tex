\documentclass{ctexart}
\usepackage{siunitx}
% \usepackage[
% paperwidth=13cm,
% paperheight=17.33cm,
% top=0.2cm,bottom=0.5cm,
% left=0.2cm,right=0.2cm,
% footskip=13pt
% ]{geometry}
\usepackage[a4paper,top=1.5cm,bottom=1.5cm,left=2cm,right=2cm]{geometry}

\usepackage{theorem}
{
	\theoremstyle{change}
	\theoremheaderfont{\bfseries}
	\theorembodyfont{\normalfont}
	\newtheorem{ti}{}[section]
}
\renewcommand{\theti}{\arabic{ti}.}

\usepackage{paralist}
\renewcommand{\theenumi}{\arabic{enumi}}
\renewcommand{\labelenumi}{(\theenumi)}

\def\hua{\uline{\hspace*{\fill}}}
\def\huahua{\uline{\hspace*{5pc}}}


\title{《计算机网络》2019-2020 第一学期考试卷\thanks{试卷编号:1920010757A}}
\author{秦淑雅}
\begin{document}
\pagestyle{plain}
\maketitle
\section{填空题(每空 $1$ 分,共 $30$ 分)}
\begin{ti}
	常见的交换技术有三种,分别为:\hua、\hua 和\hua.
\end{ti}

\begin{ti}
	TCP/IP 采用四层协议的体系结构,从高层往低层顺序为:\hua、\\\hua、\hua 和\hua.
\end{ti}

\begin{ti}
	网络适配器所实现的功能包含了:\hua 和\hua 这两个层次的功能.
\end{ti}

\begin{ti}
	与 IP 协议配套使用的三个协议是:\hua、\hua 和\hua.
\end{ti}

\begin{ti}
	集线器工作在\hua.
\end{ti}

\begin{ti}
	电磁波在 \SI{1}{KM} 电缆的传播时延约为:\hua.
\end{ti}

\begin{ti}
	因特网有两大类路由选择协议:\hua 和\hua.
\end{ti}

\begin{ti}
	运输层为\hua 之间提供端到端的逻辑通信,网络层是为\hua 之间提供逻辑通信.
\end{ti}

\begin{ti}
	TCP 传送的数据单位协议是\hua,UDP 传送的数据单位协议是\hua.
\end{ti}

\begin{ti}
	以太网规定了最短有效帧长为\hua 字节.
\end{ti}

\begin{ti}
	发送邮件的协议:\hua,读取邮件的协议:\hua.
\end{ti}

\begin{ti}
	802.11 是无线以太网的标准,在 MAC 层使用\hua,凡使用 802.11 系列的局域网称为\hua.
\end{ti}

\begin{ti}
	Internet 的本地接入中,采用拨号方式时使用的传输介质是\huahua,采用有限电视网方式时使用的传输介质是\hua.
\end{ti}

\begin{ti}
	将数字数据调制成模拟信号进行传输,通常有三种基本方式:\hua、\hua 和\hua.
\end{ti}

\section{简答题(共 $30$ 分)}
\begin{ti}
	请具体描述计算机网络体系结构.(10分)
\end{ti}

\begin{ti}
	请描述计算机网络中的时延.(8分)
\end{ti}

\begin{ti}
	名词解释 VPN、CIDR、MTU、RIP、RTT、OSI/RM.(6分)
\end{ti}

\begin{ti}
	请描述路由器的构成.(6分)
\end{ti}

\section{计算题(共 $40$ 分)}
\begin{ti}
	设收到的信息码字为 110111,检查和 CRC 为 1011,生成多项式为:$G(x) = X^{4} + X^{3} + 1$,请问收到的信息有错吗,为什么?采用 CRC 检验后,数据链路层的传输是否变成了可靠的传输?(12分)
\end{ti}

\begin{ti}
	已知地址块中的一个地址是 140.120.84.24/20.计算这个地址块中的最小地址和最大地址.地址掩码是什么?地址块中共有多少个地址?相当于多少个 C 类地址?(10分)
\end{ti}

\begin{ti}
	以下的地址前缀中的哪一个地址与 2.52.90.140 匹配?请写出计算过程.(8分)

	\begin{inparaenum}
		\item 0/4;
		\item 32/4;
		\item 4/6;
		\item 80/4
	\end{inparaenum}
\end{ti}

\begin{ti}
	设某路由器建立了如下路由表.(10分)
	\begin{center}
		\begin{tabular}{|c|c|c|}
			\hline
			目的网络 & 子网掩码 & 下一跳 \\
			\hline
			130.112.50.0 & 255.255.255.0 & R\textsubscript{0} \\
			\hline
			160.35.30.128 & 255.255.255.128 & R\textsubscript{1} \\
			\hline
			156.104.68.0 & 255.255.255.0 & R\textsubscript{2} \\
			\hline
			192.22.163.0 & 255.255.255.192 & R\textsubscript{3} \\
			\hline
			*(默认) & --- & R\textsubscript{4} \\
			\hline
		\end{tabular}
	\end{center}
	现共收到 2 个分组,其目的站 IP 地址分别为:
	\begin{enumerate}
		\item 156.104.68.112
		\item 172.63.200.24
	\end{enumerate}
	试分别计算其下一跳,并写出计算过程.
\end{ti}
\end{document}