\documentclass{ctexart}
\usepackage{amsmath,amsthm,siunitx}
\usepackage{unicode-math}
\setmainfont{XITS}
\setmathfont{XITS Math}

\setCJKmainfont{Source Han Serif CN}[
  UprightFont    = *-Regular,
  BoldFont       = *-Bold,
  ItalicFont     = 方正新楷体简体,
  BoldItalicFont = *-Bold,
  Mapping=fullwidth-stop
]
\usepackage{sourcesanspro,sourcecodepro}
\usepackage{xeCJKfntef}
\usepackage[a4paper,margin=1in]{geometry}
\theoremstyle{definition}
\newtheorem{ti}{}[section]
\def\theti{\arabic{ti}}
\def\hua#1{\CJKunderline*{#1}}

\title{《DSP 原理及应用》2018-2019第二学期考试卷\thanks{试卷编号:1819020616B}}
\author{秦淑雅}
\begin{document}
	\pagestyle{plain}
	\maketitle
	\section{填空题($40$ 分)}
	\begin{ti}
		配置 IO 口工作于外设功能或数字 IO 功能的寄存器是\hua{GPxMUX},复位时所有 GPIO 配制成\hua{数字 IO }功能状态;配置 IO 口方向的寄存器是\hua{ GPxDIR},复位时所有 GPIO 为\hua{输入}(输入/输出)状态。
	\end{ti}

	\begin{ti}
		X2812xDSP 的中断向量表地址由\hua{ VMAP}、\hua{M0M1MAP}、\hua{MP/$\overline{\text{MC}}$}、\hua{ENPIE }信号控制。
	\end{ti}

	\begin{ti}
		复位时\hua{ XF\_$\overline{\text{XPLLDIS}}$ }引脚被采样为低电平,锁相环被禁止;\hua{$\overline{\text{T1CTRIP\_PDPINTA}}$ }引脚是功率保护引脚,下降沿引发功率驱动保护中断将 EVA 的 PWM 输出引脚置为高阻态。
	\end{ti}

	\begin{ti}
		T1 的 TMS320X281X 系列 DSP 为了保护关键寄存器,在对这些特殊寄存器改写之前要执行汇编指令“\texttt{asm(“\hua{EALLOW}”)}”以置位 ST1 的 D6 位,设置寄存器执行之后要执行“\texttt{asm(“\hua{EDIS}”)}”以清除 ST1 的 D6 位;这些需要保护的特殊功能寄存器是\hua{ DSP 仿真寄存器}、\hua{Flash 寄存器}、\hua{CSM 寄存器}、\hua{PIE 中断向量表}、\hua{系统控制寄存器}、\hua{GPIO\_MUX 寄存器}、\hua{某些 eCAN 寄存器}。
	\end{ti}

	\begin{ti}
		通用定时器的比较单元产生高有效的 PWM 对称波形时占空比公式为\underline{ $\alpha = \frac{\text{TxPR}-\text{TxCMPR}}{\text{Tx\-PR}}$}。
	\end{ti}

	\begin{ti}
		定期“喂狗”实际就是周期性向\hua{复位密钥}寄存器写入\hua{0x55} $+$ \hua{0xAA}。
	\end{ti}

	\begin{ti}
		记录引脚电平跳变时刻可以用事件管理器的\hua{捕获}单元。
	\end{ti}

	\begin{ti}
		语句“\hua{\texttt{\# pragma CODE\_SECTION(AdcRegs,AdcRegsFile)}}”将 ADC 的寄存器变量 AdcRegs 定位到 AdcRegsFile 段中。
	\end{ti}

	\begin{ti}
		A/D 初始化函数文件名为\hua{\texttt{DSP28\_Adc.c}};CPU 定时器配置函数为\hua{\texttt{void ConfigCpuTimer()}}。
	\end{ti}

	\begin{ti}
		可执行文件后缀是\hua{ *.out},链接命令文件后缀是\hua{ *.cmd}。
	\end{ti}

	\begin{ti}
		定时器比较匹配事件时 TxPWM/TxCMP 引脚由低电平跳变到高电平则该引脚的输出极性模式为\hua{高有效}。
	\end{ti}

	\begin{ti}
		使能捕获单元 1 和 2,需要写指令 \texttt{EvaRegs.\hua{CAPCONA}.\hua{bit}.CAP12PN=1}。
	\end{ti}

	\begin{ti}
		TMS320X2812 扩展片外数据存储器选择 XINTF6 区,起始地址是\hua{\texttt{0X10 0000}},存储器片选信号与 DSP 的\hua{ $\overline{\text{XZCS6AND7}}$ }引脚相连接。
	\end{ti}

	\begin{ti}
		为使外设中断被响应后 PIE 控制器能响应同组的其他中断要对\hua{ PIEACK }的相关位进行手动复位,即对相应位写\hua{ 1}。
	\end{ti}

	\begin{ti}
		\hua{\texttt{DSP281x\_PieCtrl} }文件中有一个函数\hua{\texttt{InitPieCtrl(void)}(函数名)}实现对外设中断扩展模块 PIE 控制寄存器进行初始化。
	\end{ti}

	\begin{ti}
		INT1.5 是\hua{ XINT2 }中断。
	\end{ti}

	\begin{ti}
		设置全比较单元引脚输出极性的寄存器为\hua{比较方式控制寄存器}。
	\end{ti}

	\section{简答题($10 \times 3$)分}
	\begin{ti}
		使捕获单元工作需要进行什么设置?详细说明 CAPFIFOA 的 D9D8 位作用。
		\P245
	\end{ti}

	\begin{ti}
		已知使用的晶体振荡器频率,需要设置哪些寄存器,确定通用定时器的时钟基准(定时器计数一个节拍的时钟周期)。\P183--192
	\end{ti}

	\begin{ti}
		使通用定时器 T1、T2 同步的设置步骤。\P195
	\end{ti}

	\section{编程题($10 + 20$ 分)}
	\begin{ti}
		外部晶振频率为 \SI{30}{MHz},希望得到 SYSCLKOUT 为 \SI{150}{MHz},高速外设时钟为 \SI{75}{MHz},低速外设时钟为 \SI{37.5}{MHz},禁止看门狗使用 EVA、ADC 以及 SPI 外设,写出系统初始化程序。
		\begin{verbatim}
			#include"_________"
			/*功能:对 F2812 系统控制寄存器初始化
			入口参数:无
			出口参数:无*/
		\end{verbatim}
		\P43
	\end{ti}

	\begin{ti}
		使用通用定时器 1 每隔 \SI{1}{ms} 发生一次中断,在 GPIOB0 引脚上产生周期 \SI{0.8}{s} 的方波(400 次中断电平跳变一次);在高速外设时钟为 \SI{75}{MHz} 的设定下,分模块写出通用定时器 1 的初始化设置函数,系统初始化调用和中断初始化设置,引脚初始化函数,中断服务函数。
		\P210--211
	\end{ti}
\end{document}