\documentclass[电力电子]{subfiles}
\begin{document}
	\section{判断题}
	\begin{ti}
		三相全控桥式整流电路带电动机负载时,当控制角移到 $90^\circ$ 以后即进入逆变工作状态。\dui
	\end{ti}

	\begin{ti}
		晶闸管逆变电路在工作过程中,若某一晶闸管发生断路,就会造成逆变颠覆。\dui
	\end{ti}

	\begin{ti}
		电力场效应管是理想的电流控制器件。\cuo
	\end{ti}

	\begin{ti}
		电力场效应管 MOSEFET 在使用时要防止静电击穿。\dui
	\end{ti}

	\begin{ti}
		绝缘栅双极型晶体管内部为四层结构。\dui
	\end{ti}

	\begin{ti}
		斩波器属于直流/直流变换。\dui
	\end{ti}

	\begin{ti}
		以电力晶体管组成的斩波器适用于特大容量的场合。\cuo
	\end{ti}

	\begin{ti}
		斩波器用于直流电动机调速时,可将直流电源断续加到电动机上,通过通、断的时间变化来改变电压的平均值,从而改变直流电动机的转速。\dui
	\end{ti}

	\begin{ti}
		采用相位控制的交流调压电路输出为缺角正弦波,其谐波分量大。\dui
	\end{ti}

	\begin{ti}
		变流电路产生有源逆变时,晶闸管的电流方向与整流时相反。\cuo
	\end{ti}

	\begin{ti}
		可控整流电路带大电感负载时,其输出电压的波形一定与电阻负载相同。\cuo
	\end{ti}

	\begin{ti}
		当有源逆变电路负载端的电动势 $E$ 与电流方向一致时,才可实现逆变。\dui
	\end{ti}

	\begin{ti}
		绝缘栅双极型晶体管属于电流控制元件。\cuo
	\end{ti}

	\begin{ti}
		单相半桥逆变器(电压型)的输出电压为正弦波。\cuo
	\end{ti}

	\begin{ti}
		在并联谐振式晶闸管逆变器中,负载两端电压是正弦波电压,负载两端电流是正弦波电流。\cuo
	\end{ti}

	\begin{ti}
		晶闸管斩波器的作用是把可调的直流电压变为固定的直流电压。\cuo
	\end{ti}

	\begin{ti}
		把直流变交流的电路称为变频电路。\cuo
	\end{ti}

	\begin{ti}
		双向晶闸管的额定电流和普通晶闸管额定电流是有区别的,前者采用有效值,而后者采用平均值。\dui
	\end{ti}

	\begin{ti}
		三相三线交流调压电路的触发脉冲应采用宽脉冲($> 60^\circ$)或双脉冲。\dui
	\end{ti}

	\begin{ti}
		无续流二极管的可控整流电路带大电感负载时,晶闸管的导通角与控制角的大小无关。\dui
	\end{ti}

	\begin{ti}
		逆变器中 GTR、IGBT 等管子上反并联一个二极管,其作用是提供向电源反馈能量的通道。\dui
	\end{ti}

	\begin{ti}
		电力场效应管属于双极型器件。\cuo
	\end{ti}

	\begin{ti}
		电压型逆变器适用于不经常起动、制动和反转的拖动装置中。\dui
	\end{ti}

	\begin{ti}
		在电压型逆变器中,是用大电感来缓冲无功能量的。\cuo
	\end{ti}

	\begin{ti}
		无续流二极管的可控整流电路带大电感负载时,其移相范围均为 $0^\circ \sim 90^\circ$,与电路形式无关。\dui
	\end{ti}

	\begin{ti}
		有续流二极管的可控整流电路,其输出电压的波形与负载的性质无关。\dui
	\end{ti}

	\begin{ti}
		当有源逆变电路负载端的电动势 $E$ 与电流方向一致时,才可实现逆变。\dui
	\end{ti}

	\begin{ti}
		绝缘栅双极型晶体管属于电流控制元件。\cuo
	\end{ti}

	\begin{ti}
		采用相位控制的交流调压电路输出为缺角正弦波,其谐波分量大。\dui
	\end{ti}

	\begin{ti}
		晶闸管逆变电路在工作过程中,若某一晶闸管发生断路,就会造成逆变颠覆。\dui
	\end{ti}

	\begin{ti}
		在半控桥整流带大电感负载不加续流二极管电路中,电路出故障时会出现失控现象。\dui
	\end{ti}

	\begin{ti}
		在用两组反并联晶闸管的可逆系统,使直流电动机实现四象限运行时,其中一组逆变器工作在整流状态,那么另一组就工作在逆变状态。\cuo
	\end{ti}

	\begin{ti}
		晶闸管串联使用时,必须注意均流问题。\cuo
	\end{ti}

	\begin{ti}
		逆变角太大会造成逆变失败。\cuo
	\end{ti}

	\begin{ti}
		并联谐振逆变器必须是略呈电容性电路。\dui
	\end{ti}

	\begin{ti}
		给晶闸管加上正向阳极电压它就会导通。\cuo
	\end{ti}

	\begin{ti}
		有源逆变指的是把直流电能转变成交流电能送给负载。\cuo
	\end{ti}

	\begin{ti}
		在单相全控桥整流电路中,晶闸管的额定电压应取 $U_{2}$。\cuo
	\end{ti}

	\begin{ti}
		在三相半波可控整流电路中,电路输出电压波形的脉动频率为 \SI{300}{Hz}。\cuo
	\end{ti}

	\begin{ti}
		变频调速实际上是改变电动机内旋转磁场的速度达到改变输出转速的目的。\dui
	\end{ti}

	\begin{ti}
		在半控桥整流带大电感负载不加续流二极管电路中,电路出故障时会出现失控现象。\dui
	\end{ti}

	\begin{ti}
		在用两组反并联晶闸管的可逆系统,使直流电动机实现四象限运行时,其中一组逆变器工作在整流状态,那么另一组就工作在逆变状态。\cuo
	\end{ti}

	\begin{ti}
		逆变角太大会造成逆变失败。\cuo
	\end{ti}

	\begin{ti}
		并联谐振逆变器必须是略呈电容性电路。\cuo
	\end{ti}

	\begin{ti}
		给晶闸管加上正向阳极电压它就会导通。\cuo
	\end{ti}

	\begin{ti}
		在三相半波可控整流电路中,电路输出电压波形的脉动频率为 \SI{300}{Hz}。\cuo
	\end{ti}

	\begin{ti}
		双向晶闸管额定电流的定义,与普通晶闸管的定义相同。\cuo
	\end{ti}

	\begin{ti}
		触发普通晶闸管的触发脉冲,也能触发可关断晶闸管。\cuo
	\end{ti}

	\begin{ti}
		变频调速装置是属于无源逆变的范畴。\cuo
	\end{ti}

	\begin{ti}
		变流装置其功率因数的高低与电路负载阻抗的性质,无直接关系。\cuo
	\end{ti}

	\begin{ti}
		逆变失败,是因主电路元件出现损坏,触发脉冲丢失,电源缺相,或是逆变角太小造成的。\dui
	\end{ti}

	\begin{ti}
		并联与串联谐振式逆变器属于负载换流方式,无需专门换流关断电路。\cuo
	\end{ti}

	\begin{ti}
		三相半波可控整流电路,不需要用大于 $60^\circ$ 小于 $120^\circ$ 的宽脉冲触发,也不需要相隔 $60^\circ$ 的双脉冲触发,只用符合要求的相隔 $120^\circ$ 的三组脉冲触发就能正常工作。\dui
	\end{ti}

	\begin{ti}
		有源逆变装置是把逆变后的交流能量送回电网。\dui
	\end{ti}

	\begin{ti}
		三相桥式半控整流电路,带大电感性负载,有续流二极管时,当电路出故障时会发生失控现象。\cuo
	\end{ti}

	\begin{ti}
		供电电源缺相、逆变桥元件损坏、逆变换流失败等故障,也会引起逆变失败。\dui
	\end{ti}

	\begin{ti}
		三相桥式全控整流电路,输出电压波形的脉动频率是 \SI{150}{Hz}。\cuo
	\end{ti}

	\begin{ti}
		在普通晶闸管组成的全控整流电路中,带电感性负载,没有续流二极管时,导通的晶闸管在电源电压过零时不关断。\dui
	\end{ti}

	\begin{ti}
		用多重逆变电路或多电平逆变电路,可以改善逆变电路的输出波形,使它更接近正弦波。\dui
	\end{ti}

	\begin{ti}
		电压型逆变电路,为了反馈感性负载上的无功能量,必须在电力开关器件上反并联反馈二极管。\dui
	\end{ti}
\end{document}